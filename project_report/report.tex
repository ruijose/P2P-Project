%%%%%%%%%%%%%%%%%%%%%%%%%%%%%%%%%%%%%%%%%%%%%%%%%%%%%%%%%%%%%%%%%%%%%%
% LaTeX Example: Project Report
%
% Source: http://www.howtotex.com
%
% Feel free to distribute this example, but please keep the referral
% to howtotex.com
% Date: March 2011 
% 
%%%%%%%%%%%%%%%%%%%%%%%%%%%%%%%%%%%%%%%%%%%%%%%%%%%%%%%%%%%%%%%%%%%%%%
% How to use writeLaTeX: 
%
% You edit the source code here on the left, and the preview on the
% right shows you the result within a few seconds.
%
% Bookmark this page and share the URL with your co-authors. They can
% edit at the same time!
%
% You can upload figures, bibliographies, custom classes and
% styles using the files menu.
%
% If you're new to LaTeX, the wikibook is a great place to start:
% http://en.wikibooks.org/wiki/LaTeX
%
%%%%%%%%%%%%%%%%%%%%%%%%%%%%%%%%%%%%%%%%%%%%%%%%%%%%%%%%%%%%%%%%%%%%%%
% Edit the title below to update the display in My Documents
%\title{Project Report}
%
%%% Preamble
\documentclass[paper=a4, fontsize=11pt]{scrartcl}
\usepackage[T1]{fontenc}
\usepackage{fourier}
\usepackage{listings}
\usepackage[english]{babel}                             % English language/hyphenation
\usepackage[protrusion=true,expansion=true]{microtype}  
\usepackage{amsmath,amsfonts,amsthm} % Math packages
\usepackage[pdftex]{graphicx} 
\usepackage{url}
\usepackage{subcaption}
\usepackage{mathtools}
\usepackage{indentfirst}

%%% Custom sectioning
\usepackage{sectsty}
\allsectionsfont{\centering \normalfont\scshape}


%%% Custom headers/footers (fancyhdr package)
\usepackage{fancyhdr}
\usepackage[utf8]{inputenc}
\pagestyle{fancyplain}
\fancyhead{}                      % No page header
\fancyfoot[L]{}                     % Empty 
\fancyfoot[C]{}                     % Empty
\fancyfoot[R]{\thepage}                 % Pagenumbering
\renewcommand{\headrulewidth}{0pt}      % Remove header underlines
\renewcommand{\footrulewidth}{0pt}        % Remove footer underlines
\setlength{\headheight}{13.6pt}



%%% Equation and float numbering
\numberwithin{equation}{section}    % Equationnumbering: section.eq#
\numberwithin{figure}{section}      % Figurenumbering: section.fig#
\numberwithin{table}{section}       % Tablenumbering: section.tab#


%%% Maketitle metadata
\newcommand{\horrule}[1]{\rule{\linewidth}{#1}}   % Horizontal rule

\title{
    %\vspace{-1in}  
    \usefont{OT1}{bch}{b}{n}
    \normalfont \normalsize \textsc{Instituto Superior Técnico} \\ [25pt]
    \horrule{0.5pt} \\[0.4cm]
    \huge Sistema Entre Pares e Redes Sobrepostas \\
    \horrule{2pt} \\[0.5cm]
}
\subtitle{P2PBay - Peer to Peer auctions\\\vspace{5mm}\small Grupo 4}
\author{
  \normalsize Rui Pereira\\
  \normalsize \texttt{nº 70600}
  \and
  \normalsize João Domingos\\
  \normalsize \texttt{nº 73847}
  \and
  \normalsize Karan Balu\\
  \normalsize \texttt{nº 73937}
}

\date{}


%%% Begin document
\begin{document}
\maketitle
\section{Introdução}
Este relatório tem como objectivo definir o planeamento do projecto P2PBay, começando por descrever a razão pela escolha do protocolo de rede \textit{peer-to-peer}, continuando com uma explicação do que acontece quando um nó entra/sai da rede e concluindo com uma descrição da abordagem para cada função tanto para o cenário \textit{user} como para o cenário de \textit{manager}.




\section{Kademlia vs Chord vs Pastry}
Todos estes algoritmos analisados nas aulas têm como objectivo armazenar dados de forma descentralizada e escalável, sendo as pesquisas feitas \textit{<key,value>}. Após termos analisado as três implementações sugeridas no enunciado (\textit{Open Chord}, \textit{TomP2P} e \textit{FreePastry}) decidimos desenvolver o projecto com o protocolo \textit{kademlia}. Escolheu-se este protocolo uma vez que este consegue realizar queries a qualquer peer dentro de um intervalo sendo possível escolher a melhor rota com base em latência ou realizar \textit{queries} paralelas e assincronas.  Mantem a métrica de routing XOR durante todo o processo de routing enquanto que Pastry necessita de mais uma estrutura algorítmica para descobrir o \textit{peer} entre os \textit{peers} que partilham um mesmo prefixo mas o próximo b-bit digito é diferente, aumentando assim o custo de \textit{bootstrapping} e \textit{maintenance}.\par Sendo que o que se pretende implementar é um sistema de leilões onde a comunicação entre os peers deve ser o mais eficiente possível para manter a informação actualizada, decidiu-se usar a \textit{framework} \textit{TomP2P}. 



\section{Node Join Process}
Inicialmente apenas existe um \textit{Master Peer} que no nosso caso será o sigma.ist.utl.pt, este \textit{well known peer} é considerado como um  \textit{entry point} da rede. Quando um novo nó se quiser juntar à rede, liga-se em primeiro lugar ao \textit{Master Peer} onde recebe informações de configuração como os nós presentes na rede. Este usa esta informação para se ligar ao nós restantes.



\section{Node leave Process}
Quando um nó sai da rede, seja por logout ou seja porque houve uma falha, terá sempre todo o conteúdo que possui replicado nos nós mais próximos. Desta forma quando o utilizador voltar a efectuar o login numa outra máquina, conseguirá ter acesso a registos de bids, registos de compras efectuadas e items que colocou à venda. 

\section{P2PBay - Usage scenario}

\subsection{Login}

Para efectuar o login no sistema do P2PBay, um utilizador providenciará um \textit{username} e uma \textit{password} sendo que nosso sistema vai considerar uma lista de \textit{usernames} pré-definidos e caso o \textit{username} que o utilizador digitou corresponder a um dos elementos na lista, este será reencaminhado para um menu onde terá accesso às funções descritas nas próximas secções.


\subsection{Offer item to sale}
Para permitir que um utilizador venda um item, será necessário criar uma class \textit{itemSimple} cujo o construtor recebe um título, uma descrição, uma lista de objectos do tipo Bid que contem como argumentos um bid value, um peerID e um boolean para indicar se o item foi vendido ou não. De seguida associa-se uma chave a este objecto que será um hash do título do item.



\subsection{Accept bid}
Quando uma \textit{bid} for aceite pelo vendedor o \textit{item} deixa de ficar disponível para os outros poderem licitar. Numa classe \textit{itemSimple}, já previamente mencionada, actualiza-se o valor do \textit{boolean} chamado \textit{`sold`} para \textit{true} . Desta forma é possível saber quais os \textit{items} que já foram vendidos, impedindo assim que  outros peers façam bid.\par Finalmente é criada uma nova instãncia de item cujo o construtor contem o título, conteúdo em bytes e o valor da compra, sendo este objecto enviado para o peer que efectuo a compra. 



\subsection{Search item}
Para procurar um item um utilizador poderá usar apenas uma palavra como chave de procura, ou então usar operações booleanas como \textit{and}, \textit{or} e
\textit{not}. Por exemplo, procurar pelo item que contenha as duas palavras `case \textit{and}  iphone` irá devolver todos os items que contenham a palavra `case` e a palavra `iphone`. Pensamos que a solução mais correcta a usar, será o uso de \textit{indexes}, em que indexamos os items por partes do seu nome. Por cada parte usamos como valor todos os utilizadores que contenham essa parte, no nome de um item que tenham posto à venda. 

Por exemplo os utilizadores X,Y e Z puseram cada um, um item a venda:
\begin{itemize}
\item Utilizador X pôs o item 'Iphone Case' à venda
\item Utilizador Y pôs o item 'Iphone 5S' à venda
\item Utilizador Z pôs o item 'Samsung Case' à venda
\end{itemize}

A tabela de \textit{indexes} será a seguinte:
\begin{table}[h]
\centering
\begin{tabular}{|l|l|}
\hline
Key     & Value  \\ \hline
Samsung & Z      \\ \hline
Iphone  & X, Y \\ \hline
Case    & Z, X\\ \hline
5S    & Y \\ \hline
\end{tabular}
\end{table}

\subsection{Bid on item}
Um utilizador que queira efectuar uma \textit{bid} deverá previamente procurar o \textit{item} ao qual deseja fazer uma \textit{bid} obtendo desta forma, também o valor da última licitação. A única restrição em relação às licitações é o utilizador ter sempre de licitar com um valor superior à \textit{bid}
actual do \textit{item}. Não há qualquer limitação em relação ao valor da licitação desde que cumpra a regra anterior descrita. O peer cria uma nova instância de bid onde coloca como argumentos o valor da bid e o seu \textit{peer Address}, faz \textit{append} deste objecto à lista e volta a disponibilizar o objecto \textit{itemSimple} na rede. Caso um peer não consiga fazer o upload do objecto itemSimple porque dois peers tentaram fazer o bid, o peer volta a realizar o upload passado um tempo curto aleatório.
 

\subsection{View details on item} 
Para obter toda a informação acerca de um determinado \textit{item}, mais uma vez é introduzida a classe \textit{itemSimple}. Este irá conter conter certas estruturas que irão conter todos os dados acerca do \textit{item}. Estas serão enunciadas de seguida.
\begin{itemize}
  \item \textit{description}: Breve descrição acerca do item.
  \item \textit{name}: Nome do \textit{item}.
  \item \textit{allBidders}: Uma lista de todos os utilizadores que fizeram uma \textit{bid} no \textit{item}.
  \item \textit{dealer}: O nome do utilizador que pôs o \textit{item} à venda.  
\end{itemize}


\subsection{View purchase/bidding history}
De modo a fornecer ao utilizador um histórico de todas as suas compras e bids, irá estar disponível numa classe chamada \textit{user} uma lista que vai conter todos os \textit{items} comprados e uma outra lista com todos os bids efectuados. O utilizador poderá consultar estas listas após ter efectuado \textit{login} na aplicação.

\section{P2PBay - Management scenario}

\subsection{Number of nodes running P2PBay}
O número de nós que iremos usar é relevante devido ao facto de um maior número de nós proporcionar uma maior aproximação à realidade, e assim verificar com maior fiabilidade a \textit{performance} do nosso algoritmo. Também quanto maior for o número de nós, maior é a capacidade de ter utilizadores activos, tendo em conta que em cada nó, apenas pode estar um utilizador activo.
Para descobrir o número de nós activos, iremos implementar o protocolo \textit{Gossip} no nosso projecto.
\subsection{Number of users registered}
O número de utilizadores será guardado localmente em cada nó, ou seja, quando um nó entra na rede, já conhece \textit{a priori} todos os utilizadores registados.
\subsection{Number of items on sale}
O número de items à venda afecta de modo que quanto maior for o seu número, maior é o custo de manutenção dos \textit{indexes} a usar na procura de items. Sempre quando é criado um item, o peer incrementa uma variável denominada numItems e disponibiliza na rede. 


\end{document}